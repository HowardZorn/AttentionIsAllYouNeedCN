\documentclass{article}

% if you need to pass options to natbib, use, e.g.:
 \PassOptionsToPackage{numbers, compress}{natbib}
% before loading nips_2017
%
% to avoid loading the natbib package, add option nonatbib:
% \usepackage[nonatbib]{nips_2017}

% Keep final for now to remove annoying line numbers.
%\usepackage[final]{nips_2017}
% Keep final for now to remove annoying line numbers.


% to compile a camera-ready version, add the [final] option, e.g.:
\usepackage[final]{nips_2017}

\usepackage[utf8]{inputenc} % allow utf-8 input
\usepackage[UTF8]{ctex}
\usepackage[T1]{fontenc}    % use 8-bit T1 fonts
\usepackage{hyperref}       % hyperlinks
\usepackage{url}            % simple URL typesetting
\usepackage{booktabs}       % professional-quality tables
\usepackage{amsfonts}       % blackboard math symbols
\usepackage{amsmath}
\usepackage{nicefrac}       % compact symbols for 1/2, etc.
\usepackage{microtype}      % microtypography
\usepackage{subfiles}
\usepackage{xcolor}
\usepackage{multirow}
\usepackage{enumerate}
\usepackage{subfiles}
\usepackage{multirow}
\usepackage{graphicx}
\usepackage{subfiles}


% % Custom Commands:
% \newcommand\blfootnote[1]{%
%   \begingroup
%   \renewcommand\thefootnote{}\footnote{#1}%
%   \addtocounter{footnote}{-1}%
%   \endgroup
% }

\newcommand\todo[1]{\textcolor{red}{[[#1]]}}
\newcommand\mc[1]{\mathcal{#1}}
\newcommand*\samethanks[1][\value{footnote}]{\footnotemark[#1]}
%keys for memory and values. Can be changed if needed
\newcommand{\kq}{q}
\newcommand{\km}{k}
\newcommand{\vq}{o}
\newcommand{\vm}{m}
\newcommand{\Wkq}{W_q}
\newcommand{\Wkm}{W_k}
\newcommand{\Wvq}{W_o}
\newcommand{\Wvm}{W_m}
\newcommand{\dmodel}{d_{\text{model}}}
\newcommand{\dffn}{d_{\text{ffn}}}
\newcommand{\dff}{d_{\text{ff}}}
\newcommand{\mbf}[1]{\mathbf{#1}}
%\newcommand{\kq}{{q}_k}
%\newcommand{\km}{{m}_k}
%\newcommand{\vq}{{q}_v}
%\newcommand{\vm}{{m}_v}
%\newcommand{\Wkq}{{W_q}_k}
%\newcommand{\Wkm}{{W_m}_k}
%\newcommand{\Wvq}{{W_q}_v}
%\newcommand{\Wvm}{{W_m}_v}
\newcommand\concat[3]{\left[#1 \parallel_#3 #2\right]}

\title{Attention Is All You Need}

% The \author macro works with any number of authors. There are two
% commands used to separate the names and addresses of multiple
% authors: \And and \AND.
%
% Using \And between authors leaves it to LaTeX to determine where to
% break the lines. Using \AND forces a line break at that point. So,
% if LaTeX puts 3 of 4 authors names on the first line, and the last
% on the second line, try using \AND instead of \And before the third
% author name.
\author{
  \AND
  Ashish Vaswani\thanks{都是一样的贡献。这里的顺序是随机排列的。Jakob提出了用自注意力机制取代RNN,并开始努力评估这个想法。
  Ashish和Illia一起设计并实现了第一个Transformer模型,并参与了这项工作的各个方面。Noam提出了缩放点积注意力、多头注意力和无参数位置表示法,并成为几乎每一个细节都参与其中的另一个人。Niki在我们的原始代码库和tensor2tensor中设计、实现、调整和评估了无数的模型变体。Llion也尝试了新的模型变体,负责我们最初的代码库,以及高效的推理和可视化。Lukasz和Aidan花了无数个漫长的日子来设计和实现tensor2tensor的各个部分,取代了我们早期的代码库,极大地改善了结果,并大规模地加速了我们的研究。
}\\
  Google Brain\\
  \texttt{avaswani@google.com}\\
  \And
  Noam Shazeer\footnotemark[1]\\
  Google Brain\\
  \texttt{noam@google.com}\\
  \And
  Niki Parmar\footnotemark[1]\\
  Google Research\\
  \texttt{nikip@google.com}\\  
  \And
  Jakob Uszkoreit\footnotemark[1]\\
  Google Research\\
  \texttt{usz@google.com}\\
  \And  
  Llion Jones\footnotemark[1]\\
  Google Research\\
  \texttt{llion@google.com}\\   
  \And
  Aidan N. Gomez\footnotemark[1] \hspace{1.7mm}\thanks{在谷歌大脑时进行的工作}\\
  University of Toronto\\
  \texttt{aidan@cs.toronto.edu}
  \And
  {\L}ukasz Kaiser\footnotemark[1]\\
  Google Brain\\
  \texttt{lukaszkaiser@google.com}\\
  \And
  Illia Polosukhin\footnotemark[1]\hspace{1.7mm} \thanks{在谷歌研究部时进行的工作}\\
  \texttt{illia.polosukhin@gmail.com}\\  
}

\begin{document}

\maketitle

\begin{abstract}
主流的序列转换模型是基于复杂的循环或卷积神经网络,包括一个编码器和一个解码器。性能最好的模型还通过注意机制连接编码器和解码器。我们提出了一种新的简单的网络架构——Transformer,完全基于注意力机制,完全不需要循环或卷积神经网络。在两个机器翻译任务上的实验表明,这些模型在质量上更胜一筹,同时可并行性更强,所需的训练时间也大大减少。我们的模型在WMT 2014英译德任务上达到了28.4 BLEU,比现有的最佳结果(包括组合模型)提高了2 BLEU以上。在WMT 2014英译法翻译任务上,我们的模型在8个GPU上训练3.5天后(这是之前的最佳模型的训练成本的一小部分),取得了一个新的单模型最先进的BLEU得分41.8。我们通过将Transformer成功地应用到英语成分句法分析(Constituency Parsing)中,表明它能很好地泛化到其他任务中,无论是在大量的还是有限的训练数据下。
% \blfootnote{Code available at \url{https://github.com/tensorflow/tensor2tensor}}

%TODO(noam): update results for new models.

%llion@: FAIR's paper seems to concentrate solely on the convolutional aspect of their model and have the attention as an after thought almost, this gives us a good opportunity to differentiate ourselves from their paper.

%We are simpler in a number of ways and should have the simplicity as a big selling point:
%\begin{itemize}
%\item No convolutions
%\item No need for such careful initializations and %normalization.
%\item Simpler non-lineararities, they use the gated linear %units.
%\item Less layers?
%\end{itemize}
%One thing we do more is that we have self attention.
%Another selling point is the increased interpretability as %shown with the visualizations. Which comes from the %simplicity and use of only attentions.
\end{abstract}

\section{序言}

% Recurrent neural networks, long short-term memory \citep{hochreiter1997} and gated recurrent \citep{gruEval14} neural networks in particular, have been firmly established as state of the art approaches in sequence modeling and transduction problems such as language modeling and machine translation \citep{sutskever14, bahdanau2014neural, cho2014learning}. Numerous efforts have since continued to push the boundaries of recurrent language models and encoder-decoder architectures \citep{wu2016google,luong2015effective,jozefowicz2016exploring}.

循环神经网络,特别是长短期记忆(LSTM)\citep{hochreiter1997}和门控循环(GRU)\citep{gruEval14}神经网络,已经被牢固地确立为语言建模和机器翻译等序列建模和转换问题的最先进方法\citep{sutskever14, bahdanau2014neural, cho2014learning}。此后,众多的努力不断拓宽着循环神经网络语言模型和编码器-解码器架构的边界\citep{wu2016google,luong2015effective,jozefowicz2016exploring}。

% Recurrent models typically factor computation along the symbol positions of the input and output sequences. Aligning the positions to steps in computation time, they generate a sequence of hidden states $h_t$, as a function of the previous hidden state $h_{t-1}$ and the input for position $t$. This inherently sequential nature precludes parallelization within training examples, which becomes critical at longer sequence lengths, as memory constraints limit batching across examples.
%\marginpar{not sure if the memory constraints are understandable here}

RNN模型通常按输入和输出序列的符号顺序进行参数计算。在计算时,将这些位置与步长对齐,生成一个隐藏状态$h_t$的序列,作为前一个隐藏状态$h_{t-1}$和在位置$t$的输入的函数。这种固有的串行性排除了训练实例的并行化,这在较长的序列长度上变得很致命,因为内存约束限制了实例间的批处理。

% Recent work has achieved significant improvements in computational efficiency through factorization tricks \citep{Kuchaiev2017Factorization} and conditional computation \citep{shazeer2017outrageously}, while also improving model performance in case of the latter. The fundamental constraint of sequential computation, however, remains.

最近的工作通过因式化技巧\citep{Kuchaiev2017Factorization}和条件计算\citep{shazeer2017outrageously}实现了计算效率的显著提高,同时也提高了后者情况下的模型性能。然而,串行计算的基本约束仍然存在。

%\marginpar{@all: there is work on analyzing what attention really does in seq2seq models, couldn't find it right away} 

% Attention mechanisms have become an integral part of compelling sequence modeling and transduction models in various tasks, allowing modeling of dependencies without regard to their distance in the input or output sequences \citep{bahdanau2014neural, structuredAttentionNetworks}. In all but a few cases \citep{decomposableAttnModel}, however, such attention mechanisms are used in conjunction with a recurrent network.

注意机制已经成为各种任务中引人注目的序列建模和转导模型的一个组成部分,允许对依赖关系进行建模,而不考虑它们在输入或输出序列中的距离\citep{bahdanau2014neural, structuredAttentionNetworks}。然而,除了少数情况外,\citep{decomposableAttnModel},这种注意力机制都是与RNN一起使用的。

%\marginpar{not sure if "cross-positional communication" is understandable without explanation}
%\marginpar{insert exact training times and stats for the model that reaches sota earliest, maybe even a single GPU model?}

% In this work we propose the Transformer, a model architecture eschewing recurrence and instead relying entirely on an attention mechanism to draw global dependencies between input and output. The Transformer allows for significantly more parallelization and can reach a new state of the art in translation quality after being trained for as little as twelve hours on eight P100 GPUs. 

在这项工作中,我们提出了Transformer模型,一个完全摒弃了RNN的模型架构,完全依靠注意力机制来编排输入和输出之间的全局依赖关系。Transformer更加并行,并且在8个P100 GPU上训练12小时后,就可以让翻译质量达到一个新的水平。

%\marginpar{you removed the constant number of repetitions part. I wrote it because I wanted to make it clear that the model does not only perform attention once, while it's also not recurrent. I thought that might be important to get across early.}

% Just a standard paragraph with citations, rewrite.
%After the seminal papers of \citep{sutskever14}, \citep{bahdanau2014neural}, and \citep{cho2014learning}, recurrent models have become the dominant solution for both sequence modeling and sequence-to-sequence transduction. Many efforts such as \citep{wu2016google,luong2015effective,jozefowicz2016exploring} have pushed the boundaries of machine translation and language modeling with recurrent sequence models. Recent effort \citep{shazeer2017outrageously} has combined the power of conditional computation with sequence models to train very large models for machine translation, pushing SOTA at lower computational cost. Recurrent models compute a vector of hidden states $h_t$, for each time step $t$ of computation. $h_t$ is a function of both the input at time $t$ and the previous hidden state $h_t$. This dependence on the previous hidden state encumbers recurrnet models to process multiple inputs at once, and their time complexity is a linear function of the length of the input and output, both during training and inference. [What I want to say here is that although this is fine during decoding, at training time, we are given both input and output and this linear nature does not allow the RNN to process all inputs and outputs simultaneously and haven't been used on datasets that are the of the scale of the web. What's the largest dataset we have ? . Talk about Nividia and possibly other's effors to speed up things, and possibly other efforts that alleviate this, but are still limited by it's comptuational nature]. Rest of the intro: What if you could construct the state based on the actual inputs and outputs, then you could construct them all at once. This has been the foundation of many promising recent efforts, bytenet,facenet (Also talk about quasi rnn here). Now we talk about attention!! Along with cell architectures such as long short-term meory (LSTM) \citep{hochreiter1997}, and gated recurrent units (GRUs) \citep{cho2014learning}, attention has emerged as an essential ingredient in successful sequence models, in particular for machine translation. In recent years, many, if not all, state-of-the-art (SOTA) results in machine translation have been achieved with attention-based sequence models \citep{wu2016google,luong2015effective,jozefowicz2016exploring}. Talk about the neon work on how it played with attention to do self attention! Then talk about what we do.

\section{背景}

\input{background}

\section{模型结构}

\begin{figure}
  \centering
  \includegraphics[scale=0.6]{Figures/ModalNet-21}
  \caption{Transformer - 模型结构}
  \label{fig:model-arch}
\end{figure}

% Although the primary workhorse of our model is attention, 
%Our model maintains the encoder-decoder structure that is common to many so-called sequence-to-sequence models \citep{bahdanau2014neural,sutskever14}.  As in all such architectures, the encoder computes a representation of the input sequence, and the decoder consumes these representations along with the output tokens to autoregressively produce the output sequence.  Where, traditionally, the encoder and decoder contain stacks of recurrent or convolutional layers, our encoder and decoder stacks are composed of attention layers and position-wise feed-forward layers (Figure~\ref{fig:model-arch}).  The following sections describe the gross architecture and these particular components in detail.

% Most competitive neural sequence transduction models have an encoder-decoder structure \citep{cho2014learning,bahdanau2014neural,sutskever14}. Here, the encoder maps an input sequence of symbol representations $(x_1, ..., x_n)$ to a sequence of continuous representations $\mathbf{z} = (z_1, ..., z_n)$. Given $\mathbf{z}$, the decoder then generates an output sequence $(y_1,...,y_m)$ of symbols one element at a time. At each step the model is auto-regressive \citep{graves2013generating}, consuming the previously generated symbols as additional input when generating the next.

目前大部分比较热门的神经序列转换模型都是encoder-decoder结构\citep{cho2014learning,bahdanau2014neural,sutskever14}。这里,encoder将输入的一串符号表示$(x_1, ..., x_n)$映射到一串连续表示序列$\mathbf{z} = (z_1, ..., z_n)$. 对于编码得到的$\mathbf{z}$,decoder每次解码生成输出序列$(y_1,...,y_m)$的一个符号,直到生成完整的输出序列。每一步解码都是自回归的\citep{graves2013generating},也就是在生成下一个符号时将先前生成的符号作为附加输入。

% The Transformer follows this overall architecture using stacked self-attention and point-wise, fully connected layers for both the encoder and decoder, shown in the left and right halves of Figure~\ref{fig:model-arch}, respectively.

Transformer在encoder和decoder中都使用了堆叠的self-attention, point-wise和全连接层。encoder和decoder的结构分别如图~\ref{fig:model-arch}的左半部分和右半部分所示。

\subsection{Encoder和Decoder堆}

%\paragraph{Encoder:}The encoder is composed of a stack of $N=6$ identical layers. Each layer has two sub-layers. The first is a multi-head self-attention mechanism, and the second is a simple, position-wise fully connected feed-forward network.   We employ a residual connection \citep{he2016deep} around each of the two sub-layers, followed by layer normalization \cite{layernorm2016}.  That is, the output of each sub-layer is $\mathrm{LayerNorm}(x + \mathrm{Sublayer}(x))$, where $\mathrm{Sublayer}(x)$ is the function implemented by the sub-layer itself.  To facilitate these residual connections, all sub-layers in the model, as well as the embedding layers, produce outputs of dimension $\dmodel=512$.

\paragraph{Encoder:} 编码器由$N=6$个相同的层构成。每层有两个子层。第一个是一个多头的自我注意机制,第二个是一个简单的、位置上的全连接前馈网络。  我们在两个子层的每个周围采用了一个残差连接\citep{he2016deep},然后是层归一化(LayerNorm)\cite{layernorm2016}。 也就是说,每个子层的输出是$\mathrm{LayerNorm}(x + \mathrm{Sublayer}(x))$,其中$\mathrm{Sublayer}(x)$是子层本身实现的函数。为了给这些残差连接便利,模型中的所有子层以及嵌入层都会产生维度为$\dmodel=512$的输出。

%\paragraph{Decoder:}The decoder is also composed of a stack of $N=6$ identical layers.  In addition to the two sub-layers in each encoder layer, the decoder inserts a third sub-layer, which performs multi-head attention over the output of the encoder stack.  Similar to the encoder, we employ residual connections around each of the sub-layers, followed by layer normalization.  We also modify the self-attention sub-layer in the decoder stack to prevent positions from attending to subsequent positions.  This masking, combined with fact that the output embeddings are offset by one position, ensures that the predictions for position $i$ can depend only on the known outputs at positions less than $i$.

\paragraph{Decoder:} 解码器也是由$N=6$的相同层堆栈组成。除了每个编码器层的两个子层之外,解码器还插入了第三种子层,它对编码器堆的输出进行多头注意。与编码器类似,我们在每个子层周围采用残差连接,然后进行层归一化。我们还修改了解码器堆栈中的自注意子层,以防止位置关注后续位置。这种屏蔽,再加上输出嵌入总是偏移一个位置,确保对位置$i$的预测只依赖于小于$i$的位置的已知输出。

% In our model (Figure~\ref{fig:model-arch}), the encoder and decoder are composed of stacks of alternating self-attention layers (for cross-positional communication) and position-wise feed-forward layers (for in-place computation).  In addition, the decoder stack contains encoder-decoder attention layers.  Since attention is agnostic to the distances between words, our model requires a "positional encoding" to be added to the encoder and decoder input. The following sections describe all of these components in detail.

\subsection{注意力机制} \label{sec:attention}
%An attention function can be described as mapping a query and a set of key-value pairs to an output, where the query, keys, values, and output are all vectors.  The output is computed as a weighted sum of the values, where the weight assigned to each value is computed by a compatibility function of the query with the corresponding key.

注意力函数可以被描述为将一个查询(query)和一对键-值(key-value)对映射到一个输出,其中查询(query)、键(key)、值(value)和输出都是向量。输出被计算为值(value)的加权和,其中分配给每个数值的权重是由查询(query)与相应的键(key)的相似函数计算的。

\subsubsection{缩放点积注意力(Scaled Dot-Product Attention)} \label{sec:scaled-dot-prod}

% \begin{figure}
%   \centering
%   \includegraphics[scale=0.6]{Figures/ModalNet-19}
%   \caption{Scaled Dot-Product Attention.}
%   \label{fig:multi-head-att}
% \end{figure}

%We call our particular attention "Scaled Dot-Product Attention" (Figure~\ref{fig:multi-head-att}).   The input consists of queries and keys of dimension $d_k$, and values of dimension $d_v$.  We compute the dot products of the query with all keys, divide each by $\sqrt{d_k}$, and apply a softmax function to obtain the weights on the values.

我们把我们自己的独特的注意力机制称为 “缩放点积注意力”(Scaled Dot-Product Attention)(Figure~\ref{fig:multi-head-att})。 输入包括查询(query)和维度为$d_k$的键(key),以及维度为$d_v$的值(value)。 我们计算查询(query)与所有键(key)的点积,将每个键除以$\sqrt{d_k}$,并应用softmax函数来获得值(value)的权重。

%In practice, we compute the attention function on a set of queries simultaneously, packed together into a matrix $Q$.   The keys and values are also packed together into matrices $K$ and $V$.  We compute the matrix of outputs as:

在实践中,我们同时计算一组查询(query)的注意力函数,并将其打包成一个矩阵$Q$。键(key)和值(value)也被打包成矩阵$K$和$V$。我们计算输出的矩阵为:

\begin{equation}
   \mathrm{Attention}(Q, K, V) = \mathrm{softmax}\left(\frac{QK^T}{\sqrt{d_k}}\right)V
\end{equation}

%The two most commonly used attention functions are additive attention \citep{bahdanau2014neural}, and dot-product (multiplicative) attention.  Dot-product attention is identical to our algorithm, except for the scaling factor of $\frac{1}{\sqrt{d_k}}$. Additive attention computes the compatibility function using a feed-forward network with a single hidden layer.  While the two are similar in theoretical complexity, dot-product attention is much faster and more space-efficient in practice, since it can be implemented using highly optimized matrix multiplication code.

两种最常用的注意力函数是加法注意力\citep{bahdanau2014neural},和点积(乘法)注意力。 除了$\frac{1}{\sqrt{d_k}}$的比例因子外,点积注意力与我们的算法相同。加法注意力使用具有单个隐藏层的前馈网络来计算兼容性函数。 虽然两者在理论上的计算复杂度相似,但点积式注意力在实践中要快得多,而且空间效率更高,因为它可以用高度优化的矩阵乘法代码来实现。

%We scale the dot products by $1/\sqrt{d_k}$ to limit the magnitude of the dot products, which works well in practice. Otherwise, we found applying the softmax to often result in weights very close to 0 or 1, and hence minuscule gradients.

% Already described in the subsequent section
%When used as part of decoder self-attention, an optional mask function is applied just before the softmax to prevent positions from attending to subsequent positions.   This mask simply sets the logits corresponding to all illegal connections (those outside of the lower triangle) to $-\infty$.

%\paragraph{Comparison to Additive Attention: } We choose dot product attention over additive attention \citep{bahdanau2014neural} since it can be computed using highly optimized matrix multiplication code.  This optimization is particularly important to us, as we employ many attention layers in our model.

%While for small values of $d_k$ the two mechanisms perform similarly, additive attention outperforms dot product attention without scaling for larger values of $d_k$ \citep{DBLP:journals/corr/BritzGLL17}. We suspect that for large values of $d_k$, the dot products grow large in magnitude, pushing the softmax function into regions where it has extremely small gradients  \footnote{To illustrate why the dot products get large, assume that the components of $q$ and $k$ are independent random variables with mean $0$ and variance $1$.  Then their dot product, $q \cdot k = \sum_{i=1}^{d_k} q_ik_i$, has mean $0$ and variance $d_k$.}. To counteract this effect, we scale the dot products by $\frac{1}{\sqrt{d_k}}$.

虽然对于$d_k$比较小的情况,这两种机制的表现相似,但对于$d_k$比较大的情况,加法注意力优于不使用缩放的点积注意力\citep{DBLP:journals/corr/BritzGLL17}。我们怀疑,对于$d_k$比较大的情况,点积的结果的大小受到$d_k$影响而增长,将softmax函数推到它具有极小梯度的区域(译注:梯度消失)。
\footnote{为了说明为什么点积会变大,假设$q$和$k$的组成部分是独立的随机变量,均值为$0$,方差为$1$。 那么它们的点积,$q \cdot k = \sum_{i=1}^{d_k} q_ik_i$,均值为$0$,方差为$d_k$。}
为了抵消这种影响,我们用$\frac{1}{\sqrt{d_k}}$来缩放点积的结果。

%We suspect this to be caused by the dot products growing too large in magnitude to result in useful gradients after applying the softmax function.  To counteract this, we scale the dot product by $1/\sqrt{d_k}$.


\subsubsection{多头注意力机制(Multi-Head Attention)} \label{sec:multihead}

\begin{figure}
\begin{minipage}[t]{0.5\textwidth}
  \centering
  % Scaled Dot-Product Attention \\
  缩放点积注意力 \\
  \vspace{0.5cm}
  \includegraphics[scale=0.6]{Figures/ModalNet-19}
\end{minipage}
\begin{minipage}[t]{0.5\textwidth}
  \centering 
  %Multi-Head Attention \\
  多头注意力机制 \\
  \vspace{0.1cm}
  \includegraphics[scale=0.6]{Figures/ModalNet-20}  
\end{minipage}


  % \centering

  %\caption{(left) Scaled Dot-Product Attention. (right) Multi-Head Attention consists of several attention layers running in parallel.} 
  \caption{(左) 缩放点积注意力 (右) 多头注意力机制由几个并行进行的注意力层组成}
  \label{fig:multi-head-att}
\end{figure}

%Instead of performing a single attention function with $\dmodel$-dimensional keys, values and queries, we found it beneficial to linearly project the queries, keys and values $h$ times with different, learned linear projections to $d_k$, $d_k$ and $d_v$ dimensions, respectively.
%On each of these projected versions of queries, keys and values we then perform the attention function in parallel, yielding $d_v$-dimensional output values. These are concatenated and once again projected, resulting in the final values, as depicted in Figure~\ref{fig:multi-head-att}.

我们发现,用不同的、可学习的线性变换将查询、键和值分别变换到$d_k$、$d_k$和$d_v$维度上,而不是用$\dmodel$维度的键、值和查询来执行单一的注意函数。
在每一个通过上述线性变换得到的查询、键和值的基础上,我们再平行地进行注意函数,产生$d_v$维的输出值。这些值被拼接(concatenate)起来,并再次进行线性变换,从而得到最终的值,如图~\ref{fig:multi-head-att}所示。

%Multi-head attention allows the model to jointly attend to information from different representation subspaces at different positions. With a single attention head, averaging inhibits this.
多头注意力机制允许模型在不同的位置上一同关注来自不同表征子空间的信息。在单头注意力的情况下,平均化抑制了这一点。

\begin{align*}
    \mathrm{MultiHead}(Q, K, V) &= \mathrm{Concat}(\mathrm{head_1}, ..., \mathrm{head_h})W^O\\
%    \mathrm{where} \mathrm{head_i} &= \mathrm{Attention}(QW_Q_i^{\dmodel \times d_q}, KW_K_i^{\dmodel \times d_k}, VW^V_i^{\dmodel \times d_v})\\
    \text{where}~\mathrm{head_i} &= \mathrm{Attention}(QW^Q_i, KW^K_i, VW^V_i)\\
\end{align*}

%Where the projections are parameter matrices $W^Q_i \in \mathbb{R}^{\dmodel \times d_k}$, $W^K_i \in \mathbb{R}^{\dmodel \times d_k}$, $W^V_i \in \mathbb{R}^{\dmodel \times d_v}$ and $W^O \in \mathbb{R}^{hd_v \times \dmodel}$.
此处的线性变换是通过参数矩阵提供:$W^Q_i \in \mathbb{R}^{\dmodel \times d_k}$,$W^K_i \in \mathbb{R}^{\dmodel \times d_k}$, $W^V_i \in \mathbb{R}^{\dmodel \times d_v}$ 和 $W^O \in \mathbb{R}^{hd_v \times \dmodel}$.


%find it better (and no more expensive) to have multiple parallel attention layers (each over the full set of positions) with proportionally lower-dimensional keys, values and queries.  We call this "Multi-Head Attention" (Figure~\ref{fig:multi-head-att}).  The keys, values, and queries for each of these parallel attention layers are computed by learned linear transformations of the inputs to the multi-head attention.  We use different linear transformations across different parallel attention layers.  The output of the parallel attention layers are concatenated, and then passed through a final learned linear transformation. 


%In this work we employ $h=8$ parallel attention layers, or heads. For each of these we use $d_k=d_v=\dmodel/h=64$.
%Due to the reduced dimension of each head, the total computational cost is similar to that of single-head attention with full dimensionality.

在这个工作中,我们并行进行$h=8$个注意力层/头,每一个头的维度为$d_k=d_v=\dmodel/h=64$。由于每个头的维度减少,总的计算成本与全维度的单头注意力相似。

%\subsubsection{Applications of Attention in our Model}
\subsubsection{Attention在模型中的应用}

Transformer中以三种不同的方式使用了多头注意力机制:
\begin{itemize}
%  \item In "encoder-decoder attention" layers, the queries come from the previous decoder layer, and the memory keys and values come from the output of the encoder.   This allows every position in the decoder to attend over all positions in the input sequence.  This mimics the typical encoder-decoder attention mechanisms in sequence-to-sequence models such as \citep{wu2016google, bahdanau2014neural,JonasFaceNet2017}.
 \item 在“编码器-解码器注意力”("Encoder-Decoder Attention")层,queries来自先前的解码器层,并且keys和values来自编码器的输出。这使得解码器中的每个位置都能关注到输入序列中的所有位置。这与Seq2Seq模型中的经典的编码器-解码器注意力机制一致\citep{wu2016google, bahdanau2014neural,JonasFaceNet2017}。

 %\item The encoder contains self-attention layers.  In a self-attention layer all of the keys, values and queries come from the same place, in this case, the output of the previous layer in the encoder.   Each position in the encoder can attend to all positions in the previous layer of the encoder.
 \item 编码器中的自注意力层。在自注意力层中,所有的keys、values和queries都来同一个地方,这里都是来自编码器中前一层的输出。编码器中当前层的每个位置都能关注到前一层的所有位置

 %\item Similarly, self-attention layers in the decoder allow each position in the decoder to attend to all positions in the decoder up to and including that position.  We need to prevent leftward information flow in the decoder to preserve the auto-regressive property.  We implement this inside of scaled dot-product attention by masking out (setting to $-\infty$) all values in the input of the softmax which correspond to illegal connections.  See Figure~\ref{fig:multi-head-att}.
 \item 类似的,解码器中的自注意力层允许解码器中的每个位置关注当前解码位置和它前面的所有位置。这里需要屏蔽解码器中向左的信息流以保持自回归属性。具体的实现方式是在缩放后的点积注意分数中,屏蔽(设为负无穷$-\infty$)softmax的输入中所有对应着非法连接的values。见图~\ref{fig:multi-head-att}。

\end{itemize}

\subsection{逐位置前馈网络(Position-wise Feed-Forward Networks)}\label{sec:ffn}

%In addition to attention sub-layers, each of the layers in our encoder and decoder contains a fully connected feed-forward network, which is applied to each position separately and identically.  This consists of two linear transformations with a ReLU activation in between.
除了注意子层之外,我们的编码器和解码器中的每一层都包含一个全连接的前馈网络,该网络分别适用于每个位置,并且完全相同。 这包括两个线性变换,中间有一个ReLU激活。

\begin{equation}
   \mathrm{FFN}(x)=\max(0, xW_1 + b_1) W_2 + b_2
\end{equation}

%While the linear transformations are the same across different positions, they use different parameters from layer to layer. Another way of describing this is as two convolutions with kernel size 1.  The dimensionality of input and output is $\dmodel=512$, and the inner-layer has dimensionality $d_{ff}=2048$.

虽然线性变换(Linear)在不同的位置上是相同的,但它们在不同的层上使用不同的参数。另一种方式是将其作为两个内核大小为1的卷积。其输入和输出的维度为$\dmodel=512$,内层的维度为$d_{ff}=2048$。


%In the appendix, we describe how the position-wise feed-forward network can also be seen as a form of attention.

%from Jakob: The number of operations required for the model to relate signals from two arbitrary input or output positions grows in the distance between positions in input or output, linearly for ConvS2S and logarithmically for ByteNet, making it harder to learn dependencies between these positions \citep{hochreiter2001gradient}. In the transformer this is reduced to a constant number of operations, albeit at the cost of effective resolution caused by averaging attention-weighted positions, an effect we aim to counteract with multi-headed attention.


%Figure~\ref{fig:simple-att} presents a simple attention function, $A$, with a single head, that forms the basis of our multi-head attention. $A$ takes a query key vector $\kq$, matrices of memory keys $\km$ and memory values $\vm$ ,and produces a query value vector $\vq$ as 
%\begin{equation*} \label{eq:attention}
%    A(\kq, \km, \vm) = {\vm}^T (Softmax(\km \kq).
%\end{equation*}
%We linearly transform $\kq,\,\km$, and $\vm$ with learned matrices ${\Wkq \text{,} \, \Wkm}$, and ${\Wvm}$ before calling the attention function, and transform the output query with $\Wvq$ before handing it to the feed forward layer. Each attention layer has it's own set of transformation matrices, which are shared across all query positions. $A$ is applied in parallel for each query position, and is implemented very efficiently as a batch of matrix multiplies. The self-attention and encoder-decoder attention layers use $A$, but with different arguments. For example, in encdoder self-attention, queries in encoder layer $i$ attention to memories in encoder layer $i-1$. To ensure that decoder self-attention layers do not look at future words, we add $- \inf$ to the softmax logits in positions $j+1$ to query length for query position $l$.  

%In simple attention, the query value is a weighted combination of the memory values where the attention weights sum to one. Although this function performs well in practice, the constraint on attention weights can restrict the amount of information that flows from memories to queries because the query cannot focus on multiple memory positions at once, which might be desirable when translating long sequences. \marginpar{@usz, could you think of an example of this ?} We remedy this by maintaining multiple attention heads at each query position that attend to all memory positions in parallel, with a different set of parameters  per attention head $h$. 
%\marginpar{}

\subsection{嵌入和Softmax}
%Similarly to other sequence transduction models, we use learned embeddings to convert the input tokens and output tokens to vectors of dimension $\dmodel$.  We also use the usual learned linear transformation and softmax function to convert the decoder output to predicted next-token probabilities.  In our model, we share the same weight matrix between the two embedding layers and the pre-softmax linear transformation, similar to \citep{press2016using}.   In the embedding layers, we multiply those weights by $\sqrt{\dmodel}$.

与其他序列转换模型类似,我们使用可学习的嵌入将输入标记和输出标记转换为维度为$\dmodel$的向量。我们还使用通常的可学习的线性层(Linear)和softmax函数将解码器输出转换为预测的下一个标记概率。 在我们的模型中,我们在两个嵌入层和pre-softmax(在softmax前面的)线性层之间共享相同的权重矩阵,类似于\citep{press2016using}。在嵌入层中,我们把这些权重乘以$\sqrt{\dmodel}$。

\subsection{Positional Encoding}
%Since our model contains no recurrence and no convolution, in order for the model to make use of the order of the sequence, we must inject some information about the relative or absolute position of the tokens in the sequence.  To this end, we add "positional encodings" to the input embeddings at the bottoms of the encoder and decoder stacks.  The positional encodings have the same dimension $\dmodel$ as the embeddings, so that the two can be summed.   There are many choices of positional encodings, learned and fixed \citep{JonasFaceNet2017}.
由于我们的模型不包含RNN和CNN,为了使模型能够利用序列的顺序,我们必须注入一些关于序列中标记的相对或绝对位置的信息。 为此,我们在编码器和解码器堆栈的底部为输入嵌入添加 “位置编码”(Positional Encoding)。 位置编码与嵌入具有相同的维度$\dmodel$,因此两者可以相加。位置编码有很多选择,有学习的,也有固定的 \citep{JonasFaceNet2017}。


%In this work, we use sine and cosine functions of different frequencies:
在这项工作中,我们使用不同频率的正弦和余弦函数来进行Positional Encoding:

\begin{align*}
    PE_{(pos,2i)} = sin(pos / 10000^{2i/\dmodel}) \\
    PE_{(pos,2i+1)} = cos(pos / 10000^{2i/\dmodel})
\end{align*}

%where $pos$ is the position and $i$ is the dimension.  That is, each dimension of the positional encoding corresponds to a sinusoid.  The wavelengths form a geometric progression from $2\pi$ to $10000 \cdot 2\pi$.  We chose this function because we hypothesized it would allow the model to easily learn to attend by relative positions, since for any fixed offset $k$, $PE_{pos+k}$ can be represented as a linear function of $PE_{pos}$.

其中$pos$是位置,$i$是维度。也就是说,位置编码的每个维度对应于一个正弦波,其波长形成一个几何级数,从$2\pi$到$10000\cdot 2\pi$。我们选择这个函数是因为我们假设它能让模型很容易地学会通过相对位置来进行推理,因为对于任何固定的偏移量$k$,$PE_{pos+k}$可以被表示为$PE_{pos}$的线性函数。

%We also experimented with using learned positional embeddings \citep{JonasFaceNet2017} instead, and found that the two versions produced nearly identical results (see Table~\ref{tab:variations} row (E)).  We chose the sinusoidal version because it may allow the model to extrapolate to sequence lengths longer than the ones encountered during training.

我们也尝试采用可学习的positional embeddings\citep{JonasFaceNet2017}代替上面的方法,然后发现两者得到了几乎一样的结果(见表~\ref{tab:variations}(E)行)。我们选择正弦波版本是因为它可能允许模型推断出比训练期间遇到的序列长度更长的序列。
 
\section{为什么是Self-Attention}
%We focus on the general task of mapping one variable-length sequence of symbol representations ${x_1, ..., x_n} \in \mathbb{R}^d$ to another sequence of the same length ${y_1, ..., y_n} \in \mathbb{R}^d$. \marginpar{should we use this notation? alternatively we can just say "d-dimensional vectors"}

%In this section we compare various aspects of self-attention layers to the recurrent and convolutional layers commonly used for mapping one variable-length sequence of symbol representations $(x_1, ..., x_n)$ to another sequence of equal length $(z_1, ..., z_n)$, with $x_i, z_i \in \mathbb{R}^d$, such as a hidden layer in a typical sequence transduction encoder or decoder. Motivating our use of self-attention we consider three desiderata.

在本小节,我们将自我注意层的各个方面与通常用于将一个可变长度的符号表示序列$(x_1, ..., x_n)$映射到另一个等长的序列$(z_1, ..., z_n)$的RNN和CNN层进行比较,其中$x_i, z_i \in \mathbb{R}^d$,是典型的序列转换编码器或解码器中的的隐藏层。我们使用自注意力机制的动机是有下面三个方面的好处。

% One is the total computational complexity per layer.
% Another is the amount of computation that can be parallelized, as measured by the minimum number of sequential operations required.

一个是每层的总计算复杂性。
另一个是可以并行化的计算量,以所需的最小顺序操作数量来衡量。


%The third is the path length between long-range dependencies in the network. Learning long-range dependencies is a key challenge in many sequence transduction tasks. One key factor affecting the ability to learn such dependencies is the length of the paths forward and backward signals have to traverse in the network. The shorter these paths between any combination of positions in the input and output sequences, the easier it is to learn long-range dependencies \citep{hochreiter2001gradient}. Hence we also compare the maximum path length between any two input and output positions in networks composed of the different layer types.

第三是网络中长距离依赖关系之间的路径长度。学习长距离的依赖关系是许多序列转换任务中的一个关键挑战。影响学习这种依赖关系能力的一个关键因素是前向和后向信号在网络中必须穿越的路径的长度。在输入和输出序列的任何位置组合之间的这些路径越短,就越容易学习长距离的依赖关系。因此,我们还比较了由不同层类型组成的网络中任何两个输入和输出位置之间的最大路径长度。

%\subsection{Computational Performance and Path Lengths}

\begin{table}[t]
\caption{
  不同层类型的最大路径长度、每层复杂性和最小顺序操作数量。$n$是序列长度,$d$是表示维度,$k$是卷积的核大小,$r$是限制性自注意中的邻域大小。}
  %Attention models are quite efficient for cross-positional communications when sequence length is smaller than channel depth.
\label{tab:op_complexities}
\begin{center}
\vspace{-1mm}
%\scalebox{0.75}{

\begin{tabular}{lccc}
\toprule
% Layer Type & Complexity per Layer & Sequential & Maximum Path Length  \\
%            &             & Operations &   \\
层的类型 & 层的复杂度 & 顺序操作数量 &  最长路径长度  \\
\hline
\rule{0pt}{2.0ex}自注意 & $O(n^2 \cdot d)$ & $O(1)$ & $O(1)$ \\
循环神经网络 & $O(n \cdot d^2)$ & $O(n)$ & $O(n)$ \\

卷积 & $O(k \cdot n \cdot d^2)$ & $O(1)$ & $O(log_k(n))$ \\
%\cmidrule
自注意力(受限的)& $O(r \cdot n \cdot d)$ & $O(1)$ & $O(n/r)$ \\

%Convolutional (separable) & $O(k \cdot n \cdot d + n \cdot d^2)$ & $O(1)$ & $O(log_k(n))$ \\

%Position-wise Feed-Forward & $O(n \cdot d^2)$ & $O(1)$ & $\infty$ \\

%Fully Connected & $O(n^2 \cdot d^2)$ & $O(1)$ & $O(1)$ \\
%Convolutional (separable) & $O(k \cdot n \cdot d + n \cdot d^2)$ & $O(1)$ & $O(log_k(n))$ \\

%Position-wise Feed-Forward & $O(n \cdot d^2)$ & $O(1)$ & $\infty$ \\

%Fully Connected & $O(n^2 \cdot d^2)$ & $O(1)$ & $O(1)$ \\
\bottomrule
\end{tabular}
%}
\end{center}
\end{table}


%\begin{table}[b]
%\caption{
%  Maximum path lengths, per-layer complexity and minimum number of sequential operations for different layer types. $n$ is the sequence length, $d$ is the representation dimensionality, $k$ is the kernel size of convolutions and $r$ the size of the neighborhood in localized self-attention.}
  %Attention models are quite efficient for cross-positional communications when sequence length is smaller than channel depth.
%\label{tab:op_complexities}
%\begin{center}
%\vspace{-1mm}
%%\scalebox{0.75}{
%
%\begin{tabular}{lccc}
%\hline
%Layer Type & Receptive & Complexity per Layer & Sequential  %\\
%           & Field Size     &            & Operations  \\
%\hline
%Self-Attention & $n$ & $O(n^2 \cdot d)$ & $O(1)$ \\
%Recurrent & $n$ & $O(n \cdot d^2)$ & $O(n)$ \\

%Convolutional & $k$ & $O(k \cdot n \cdot d^2)$ & %$O(log_k(n))$ \\
%\hline 
%Self-Attention (localized)& $r$ & $O(r \cdot n \cdot d)$ & %$O(1)$ \\

%Convolutional (separable) & $k$ & $O(k \cdot n \cdot d + n %\cdot d^2)$ & $O(log_k(n))$ \\

%Position-wise Feed-Forward & $1$ & $O(n \cdot d^2)$ & $O(1)$ %\\

%Fully Connected & $n$ & $O(n^2 \cdot d^2)$ & $O(1)$ \\

%\end{tabular}
%%}
%\end{center}
%\end{table}

%The receptive field size of a layer is the number of different input representations that can influence any particular output representation. Recurrent layers and self-attention layers have a full receptive field equal to the sequence length $n$. Convolutional layers have a limited receptive field equal to their kernel width $k$, which is generally chosen to be small in order to limit computational cost.

%As noted in Table \ref{tab:op_complexities}, a self-attention layer connects all positions with a constant number of sequentially executed operations, whereas a recurrent layer requires $O(n)$ sequential operations.

如表\ref{tab:op_complexities}所示,一个自注意力层只需常数顺序执行的操作数量就可以连接所有位置,同时一个RNN层需要$O(n)$的顺序操作。

%In terms of computational complexity, self-attention layers are faster than recurrent layers when the sequence length $n$ is smaller than the representation dimensionality $d$, which is most often the case with sentence representations used by state-of-the-art models in machine translations, such as word-piece \citep{wu2016google} and byte-pair \citep{sennrich2015neural} representations.
%To improve computational performance for tasks involving very long sequences, self-attention could be restricted to considering only a neighborhood of size $r$ in the input sequence centered around the respective output position. This would increase the maximum path length to $O(n/r)$. We plan to investigate this approach further in future work.

就计算复杂度而言,当序列长度$n$小于表示维度$d$时,自我关注层比递归层更快,这在机器翻译中最先进的模型所使用的句子表示中是最常见的,如词片\citep{wu2016google}和字节对\citep{sennrich2015neural}表示。
为了提高涉及超长序列的任务的计算性能,可以限制自我注意,只考虑输入序列中以各自输出位置为中心的大小为$r$的邻域。这将使最大路径长度增加到$O(n/r)$。我们计划在未来的工作中进一步研究这种方法。

%A single convolutional layer with kernel width $k < n$ does not connect all pairs of input and output positions. Doing so requires a stack of $O(n/k)$ convolutional layers in the case of contiguous kernels, or $O(log_k(n))$ in the case of dilated convolutions \citep{NalBytenet2017}, increasing the length of the longest paths between any two positions in the network.
%Convolutional layers are generally more expensive than recurrent layers, by a factor of $k$. Separable convolutions \citep{xception2016}, however, decrease the complexity considerably, to $O(k \cdot n \cdot d + n \cdot d^2)$. Even with $k=n$, however, the complexity of a separable convolution is equal to the combination of a self-attention layer and a point-wise feed-forward layer, the approach we take in our model.

卷积核宽度为$k<n$的单一卷积层并不能连接所有的输入和输出位置对。这样做需要堆叠$O(n/k)$的卷积层(在连续核的情况下),或者$O(log_k(n))$在空洞卷积(扩张卷积)\citep{NalBytenet2017}的情况下,增加网络中任何两个位置之间最长路径的长度。
CNN层通常比RNN层更昂贵,代价为$k$的系数。然而,可分离卷积层的复杂度大大降低,为$O(k\cdot n\cdot d + n\cdot d^2)$。然而,即使$k=n$,可分离卷积的复杂度也等于自注意力层和逐点前馈层(FFN)的组合,也就是我们模型中采取的方法。

%\subsection{Unfiltered Bottleneck Argument}

%An orthogonal argument can be made for self-attention layers based on when the layer imposes the bottleneck of mapping all of the information used to compute a given output position into a single, fixed-length vector. ...

%There is a second argument for self-attention layers which we call the unfiltered bottleneck argument.   In both recurrent and the convolutional layers, the information that position $i$ receives from the other positions is compressed to a vector of dimension $d$ before it ever can be filtered by the content $x_i$.  More precisely, we can express $y_i = F(i, x_i, G(i, \{x_{j \neq i}\}))$, where $G(i, \{x_{j \neq i}\})$ is a vector of dimension $d$.  Intuitively, we would expect that this would cause a large amount of irrelevant information to crowd out the relevant information.  Self-attention does not suffer from the unfiltered bottleneck problem, since the aggregation happens after filtering, and so, intuitively, we have the chance of transmitting lots of relevant information.

%As side benefit, self-attention could yield more interpretable models. We inspect attention distributions from our models and present and discuss examples in the appendix. Not only do individual attention heads clearly learn to perform different tasks, many appear to exhibit behavior related to the syntactic and semantic structure of the sentences.

作为副作用,自注意力机制可以得到更多可解释的模型。我们从我们的模型中检查了注意力的分布,并在附录中提出和讨论了一些例子。不仅个别注意力头明显学会了执行不同的任务,许多头似乎表现出与句子的句法和语义结构有关的行为。


\section{训练}
%This section describes the training regime for our models. 

这个节表述了我们模型的训练方法

%In order to speed up experimentation, our ablations are performed relative to a smaller base model described in detail in Section \ref{sec:results}.

\subsection{训练数据和分批方案}
%We trained on the standard WMT 2014 English-German dataset consisting of about 4.5 million sentence pairs.  Sentences were encoded using byte-pair encoding \citep{DBLP:journals/corr/BritzGLL17}, which has a shared source-target vocabulary of about 37000 tokens. For English-French, we used the significantly larger WMT 2014 English-French dataset consisting of 36M sentences and split tokens into a 32000 word-piece vocabulary \citep{wu2016google}.  Sentence pairs were batched together by approximate sequence length.  Each training batch contained a set of sentence pairs containing approximately 25000 source tokens and 25000 target tokens. 
我们在由大约450万个句子对组成的标准WMT 2014英译德数据集上进行训练. 句子使用字节对编码\citep{DBLP:journals/corr/BritzGLL17},它有一个共享的源-目标词汇约37000个标记.对于英译法,我们使用了明显更大的WMT 2014英译法数据集,该数据集由3600万个句子组成,并将标记拆分为32000个词片(word-piece)\citep{wu2016google}. 句子对按近似的序列长度被分到一起. 每个训练批次包含一组句子对,其中包含大约25000个源标记和25000个目标标记.  

\subsection{硬件和时间安排}

%We trained our models on one machine with 8 NVIDIA P100 GPUs.  For our base models using the hyperparameters described throughout the paper, each training step took about 0.4 seconds.  We trained the base models for a total of 100,000 steps or 12 hours.  For our big models,(described on the bottom line of table \ref{tab:variations}), step time was 1.0 seconds.  The big models were trained for 300,000 steps (3.5 days).

我们在一台有8个NVIDIA P100 GPU的机器上训练我们的模型. 对于我们的基础模型,使用本文所述的超参数,每个训练步骤大约需要0.4秒. 我们总共训练了100,000步或12小时的基础模型. 对于我们的大模型,(在表\ref{tab:variations}的最后一行描述),一步的时间为1.0秒. 大模型训练了30万步(3.5天).

\subsection{优化器} 

%We used the Adam optimizer~\citep{kingma2014adam} with $\beta_1=0.9$, $\beta_2=0.98$ and $\epsilon=10^{-9}$.  We varied the learning rate over the course of training, according to the formula:

我们使用Adam优化器~\citep{kingma2014adam},参数为$\beta_1=0.9$,$\beta_2=0.98$,$\epsilon=10^{-9}$.我们在训练过程中更变学习率,如下式:

\begin{equation}
lrate = \dmodel^{-0.5} \cdot
  \min({step\_num}^{-0.5},
    {step\_num} \cdot {warmup\_steps}^{-1.5})
\end{equation}

%This corresponds to increasing the learning rate linearly for the first $warmup\_steps$ training steps, and decreasing it thereafter proportionally to the inverse square root of the step number.  We used $warmup\_steps=4000$.

这相当于在第一个$warmup\_steps$训练步骤中线性增加学习率,然后按步数的反平方根比例递减.我们设定$warmup\_steps=4000$.

\subsection{正规化} \label{sec:reg}

%We employ three types of regularization during training: 
%\paragraph{Residual Dropout} We apply dropout \citep{srivastava2014dropout} to the output of each sub-layer, before it is added to the sub-layer input and normalized.   In addition, we apply dropout to the sums of the embeddings and the positional encodings in both the encoder and decoder stacks.  For the base model, we use a rate of $P_{drop}=0.1$.
我们在训练过程中采用了三种类型的正则化.
\paragraph{残差和Dropout} 在每个子层的输出被添加到子层的输入中并被规范化之前,我们对每个子层的输出应用了dropout\citep{srivastava2014dropout}.此外,我们还对编码器和解码器堆中的嵌入和位置编码的总和应用了dropout. 对于基础模型,我们使用$P_{drop}=0.1$的比率.

% \paragraph{Attention Dropout} Query to key attentions are structurally similar to hidden-to-hidden weights in a feed-forward network, albeit across positions. The softmax activations yielding attention weights can then be seen as the analogue of hidden layer activations. A natural possibility is to extend dropout \citep{srivastava2014dropout} to attention. We implement attention dropout by dropping out attention weights as,
% \begin{equation*}
%   \mathrm{Attention}(Q, K, V) = \mathrm{dropout}(\mathrm{softmax}(\frac{QK^T}{\sqrt{d}}))V
% \end{equation*}
% In addition to residual dropout, we found attention dropout to be beneficial for our parsing experiments.  

%\paragraph{Symbol Dropout} In the source and target embedding layers, we replace a random subset of the token ids with a sentinel id.  For the base model, we use a rate of $symbol\_dropout\_rate=0.1$.  Note that this applies only to the auto-regressive use of the target ids - not their use in the cross-entropy loss. 

%\paragraph{Attention Dropout} Query to memory attentions are structurally similar to hidden-to-hidden weights in a feed-forward network, albeit across positions. The softmax activations yielding attention weights can then be seen as the analogue of hidden layer activations. A natural possibility is to extend dropout \citep{srivastava2014dropout} to attentions. We implement attention dropout by dropping out attention weights as,
%\begin{equation*}
%   A(Q, K, V) = \mathrm{dropout}(\mathrm{softmax}(\frac{QK^T}{\sqrt{d}}))V
%\end{equation*}
%As a result, the query will not be able to access the memory values at the dropped out position. In our experiments, we tried attention dropout rates of 0.2, and 0.3, and found it to work favorably for English-to-German translation.
%$attention\_dropout\_rate=0.2$.

%%\paragraph{Label Smoothing} During training, we employed label smoothing of value $\epsilon_{ls}=0.1$ \citep{DBLP:journals/corr/SzegedyVISW15}.  This hurts perplexity, as the model learns to be more unsure, but improves accuracy and BLEU score.

\paragraph{标签平滑} 在训练中,我们采用了标签平滑,$epsilon_{ls}=0.1$\citep{DBLP:journals/corr/SzegedyVISW15}. 这损害了迷惑度,因为模型更加不确定,但提高了准确性和BLEU得分.

 
\section{结果} \label{sec:results}
\subsection{机器翻译}
\begin{table}[t]
\begin{center}
\caption{在英译德和英译法的newstest2014测试中,Transformer取得了比以前最先进的模型更好的BLEU分数,而培训成本却只有一小部分。}
\label{tab:wmt-results}
\vspace{-2mm}
%\scalebox{1.0}{
\begin{tabular}{lccccc}
\toprule
\multirow{2}{*}{\vspace{-2mm}模型} & \multicolumn{2}{c}{BLEU} & & \multicolumn{2}{c}{训练成本 (FLOPs)} \\
\cmidrule{2-3} \cmidrule{5-6} 
%& EN-DE & EN-FR & & EN-DE & EN-FR \\ 
& 英译德 & 英译法 & & 英译德 & 英译法 \\ 
\hline
ByteNet \citep{NalBytenet2017} & 23.75 & & & &\\
Deep-Att + PosUnk \citep{DBLP:journals/corr/ZhouCWLX16} & & 39.2 & & & $1.0\cdot10^{20}$ \\
GNMT + RL \citep{wu2016google} & 24.6 & 39.92 & & $2.3\cdot10^{19}$  & $1.4\cdot10^{20}$\\
ConvS2S \citep{JonasFaceNet2017} & 25.16 & 40.46 & & $9.6\cdot10^{18}$ & $1.5\cdot10^{20}$\\
MoE \citep{shazeer2017outrageously} & 26.03 & 40.56 & & $2.0\cdot10^{19}$ & $1.2\cdot10^{20}$ \\
\hline
\rule{0pt}{2.0ex}Deep-Att + PosUnk 组合 \citep{DBLP:journals/corr/ZhouCWLX16} & & 40.4 & & &
 $8.0\cdot10^{20}$ \\
GNMT + RL 组合 \citep{wu2016google} & 26.30 & 41.16 & & $1.8\cdot10^{20}$  & $1.1\cdot10^{21}$\\
ConvS2S 组合\citep{JonasFaceNet2017} & 26.36 & \textbf{41.29} & & $7.7\cdot10^{19}$ & $1.2\cdot10^{21}$\\
\specialrule{1pt}{-1pt}{0pt}
\rule{0pt}{2.2ex}Transformer (基础模型) & 27.3 & 38.1 & & \multicolumn{2}{c}{\boldmath$3.3\cdot10^{18}$}\\
Transformer (大) & \textbf{28.4} & \textbf{41.8} & & \multicolumn{2}{c}{$2.3\cdot10^{19}$} \\
%\hline
%\specialrule{1pt}{-1pt}{0pt}
%\rule{0pt}{2.0ex}
\bottomrule
\end{tabular}
%}
\end{center}
\end{table}


%On the WMT 2014 English-to-German translation task, the big transformer model (Transformer (big) in Table~\ref{tab:wmt-results}) outperforms the best previously reported models (including ensembles) by more than $2.0$ BLEU, establishing a new state-of-the-art BLEU score of $28.4$.  The configuration of this model is listed in the bottom line of Table~\ref{tab:variations}.  Training took $3.5$ days on $8$ P100 GPUs.  Even our base model surpasses all previously published models and ensembles, at a fraction of the training cost of any of the competitive models.

在WMT 2014的英译德任务中,大Transformer模型(表~\ref{tab:wmt-results}中的Transformer (大))比之前报道的最佳模型(包括组合模型)多出$2.0$ BLEU,得到了新的最先进的BLEU得分$28.4$。 这个模型的配置列在表~\ref{tab:variations}的最下面一行。 在$8$个P100 GPU上,花了$3.5$天时间训练。 即使是我们的基本模型,也超过了以前发表的所有模型和组合模型,而训练成本只是任何有竞争力的模型的一小部分。

%On the WMT 2014 English-to-French translation task, our big model achieves a BLEU score of $41.0$, outperforming all of the previously published single models, at less than $1/4$ the training cost of the previous state-of-the-art model. The Transformer (big) model trained for English-to-French used dropout rate $P_{drop}=0.1$, instead of $0.3$.

在WMT 2014的英译法任务中,我们的大模型取得了$41.0$BLEU的分数,超过了之前发布的所有单一模型,而训练成本还不到之前最先进模型的$1/4$。为英译法训练的Transformer(大)模型使用了dropout率$P_{drop}=0.1$,而不是$0.3$。

% For the base models, we used a single model obtained by averaging the last 5 checkpoints, which were written at 10-minute intervals.  For the big models, we averaged the last 20 checkpoints. We used beam search with a beam size of $4$ and length penalty $\alpha=0.6$ \citep{wu2016google}.  These hyperparameters were chosen after experimentation on the development set.  We set the maximum output length during inference to input length + $50$, but terminate early when possible \citep{wu2016google}.

对于基本模型,我们使用了通过平均最后5个检查点得到的单一模型,这些检查点是以10分钟的时间间隔写入的。对于大模型,我们对最后的20个检查点进行了平均。我们使用了beam search,beam大小为$4$,长度惩罚为$\alpha=0.6$ \citep{wu2016google}。这些超参数是在开发集上进行实验后选择的。 我们将推理过程中的最大输出长度设置为输入长度+$50$,但会在可能的情况下提前终止\citep{wu2016google}。

表\ref{tab:wmt-results}总结了我们的结果,并将我们的翻译质量和训练成本与文献中的其他模型架构进行比较。 我们通过将训练时间、所使用的GPU数量和每个GPU的持续单精度浮点能力的估计值相乘来估计训练一个模型所使用的浮点运算的数量 \footnote{K80、K40、M40和P100分别为2.8、3.7、6.0和9.5 TFLOPS。}。
%where we compare against the leading machine translation results in the literature. Even our smaller model, with number of parameters comparable to ConvS2S, outperforms all existing single models, and achieves results close to the best ensemble model.

\subsection{模型变体}

\begin{table}[t]
\caption{Transformer结构的变体。未列出的数值与基本模型的数值相同。 所有指标都是在英译德开发集newstest2013上得到的。 根据我们的字节对编码,列出的困惑度是按词片(wordpiece)计算的,不应该与按词(word)计算的困惑度进行比较。}
\label{tab:variations}
\begin{center}
\vspace{-2mm}
%\scalebox{1.0}{
\begin{tabular}{c|ccccccccc|ccc}
\hline\rule{0pt}{2.0ex}
 & \multirow{2}{*}{$N$} & \multirow{2}{*}{$\dmodel$} &
\multirow{2}{*}{$\dff$} & \multirow{2}{*}{$h$} & 
\multirow{2}{*}{$d_k$} & \multirow{2}{*}{$d_v$} & 
\multirow{2}{*}{$P_{drop}$} & \multirow{2}{*}{$\epsilon_{ls}$} &
train & PPL & BLEU & params \\
 & & & & & & & & & steps & (dev) & (dev) & $\times10^6$ \\
% & & & & & & & & & & & & \\
\hline\rule{0pt}{2.0ex}
base & 6 & 512 & 2048 & 8 & 64 & 64 & 0.1 & 0.1 & 100K & 4.92 & 25.8 & 65 \\
\hline\rule{0pt}{2.0ex}
\multirow{4}{*}{(A)}
& & & & 1 & 512 & 512 & & & & 5.29 & 24.9 &  \\
& & & & 4 & 128 & 128 & & & & 5.00 & 25.5 &  \\
& & & & 16 & 32 & 32 & & & & 4.91 & 25.8 &  \\
& & & & 32 & 16 & 16 & & & & 5.01 & 25.4 &  \\
\hline\rule{0pt}{2.0ex}
\multirow{2}{*}{(B)}
& & & & & 16 & & & & & 5.16 & 25.1 & 58 \\
& & & & & 32 & & & & & 5.01 & 25.4 & 60 \\
\hline\rule{0pt}{2.0ex}
\multirow{7}{*}{(C)}
& 2 & & & & & & & &            & 6.11 & 23.7 & 36 \\
& 4 & & & & & & & &            & 5.19 & 25.3 & 50 \\
& 8 & & & & & & & &            & 4.88 & 25.5 & 80 \\
& & 256 & & & 32 & 32 & & &    & 5.75 & 24.5 & 28 \\
& & 1024 & & & 128 & 128 & & & & 4.66 & 26.0 & 168 \\
& & & 1024 & & & & & &         & 5.12 & 25.4 & 53 \\
& & & 4096 & & & & & &         & 4.75 & 26.2 & 90 \\
\hline\rule{0pt}{2.0ex}
\multirow{4}{*}{(D)}
& & & & & & & 0.0 & & & 5.77 & 24.6 &  \\
& & & & & & & 0.2 & & & 4.95 & 25.5 &  \\
& & & & & & & & 0.0 & & 4.67 & 25.3 &  \\
& & & & & & & & 0.2 & & 5.47 & 25.7 &  \\
\hline\rule{0pt}{2.0ex}
(E) & & \multicolumn{7}{c}{positional embedding instead of sinusoids} & & 4.92 & 25.7 & \\
\hline\rule{0pt}{2.0ex}
big & 6 & 1024 & 4096 & 16 & & & 0.3 & & 300K & \textbf{4.33} & \textbf{26.4} & 213 \\
\hline
\end{tabular}
%}
\end{center}
\end{table}


%Table \ref{tab:ende-results}. Our base model for this task uses 6 attention layers, 512 hidden dim, 2048 filter dim, 8 attention heads with both attention and symbol dropout of 0.2 and 0.1 respectively. Increasing the filter size of our feed forward component to 8192 increases the BLEU score on En $\to$ De by $?$. For both the models, we use beam search decoding of size $?$ and length penalty with an alpha of $?$ \cite? \todo{Update results}

%To evaluate the importance of different components of the Transformer, we varied our base model in different ways, measuring the change in performance on English-to-German translation on the development set, newstest2013. We used beam search as described in the previous section, but no checkpoint averaging.  We present these results in Table~\ref{tab:variations}.  

为了评估Transformer不同组件的重要性,我们以不同的方式改变了我们的基础模型,在开发集newstest2013上测量英德翻译的性能变化。我们使用了上一节中描述的beam search,但没有使用检查点平均法。 我们在表~\ref{tab:variations}中介绍了这些结果。

%In Table~\ref{tab:variations} rows (A), we vary the number of attention heads and the attention key and value dimensions, keeping the amount of computation constant, as described in Section \ref{sec:multihead}. While single-head attention is 0.9 BLEU worse than the best setting, quality also drops off with too many heads.

在表~\ref{tab:variations}行(A)中,我们改变了注意力机制头的数量以及注意力机制键(key)和值(value)的维度,保持计算量不变,如节 \ref{sec:multihead}所述。虽然单头注意力机制比最佳设置差0.9 BLEU,但随着头数的增加,翻译质量也会下降。

%In Table~\ref{tab:variations} rows (B), we observe that reducing the attention key size $d_k$ hurts model quality. This suggests that determining compatibility is not easy and that a more sophisticated compatibility function than dot product may be beneficial. We further observe in rows (C) and (D) that, as expected, bigger models are better, and dropout is very helpful in avoiding over-fitting.  In row (E) we replace our sinusoidal positional encoding with learned positional embeddings \citep{JonasFaceNet2017}, and observe nearly identical results to the base model.

在表~\ref{tab:variations}行(B)中,我们观察到,减少键(key)的维度大小$d_k$会伤害模型质量。这表明,确定相似性并不容易,比点乘更复杂的相似函数可能是有益的。我们在(C)和(D)行中进一步观察到,正如预期的那样,更大的模型更好,而且dropout对于避免过拟合非常有帮助。 在(E)行中,我们用可学习的位置嵌入代替我们的正弦波位置编码\citep{JonasFaceNet2017},观察到的结果与基础模型几乎相同。

%To evaluate the importance of different components of the Transformer, we use our base model to ablate on a single hyperparameter at each time and measure the change in performance on English$\to$German translation. Our variations in Table~\ref{tab:variations} show that the number of attention layers and attention heads is the most important architecture hyperparamter However, the we do not see performance gains beyond 6 layers, suggesting that we either don't have enough data to train a large model or we need to turn up regularization. We leave this exploration for future work. Among our regularizers, attention dropout has the most significant impact on performance.


%Increasing the width of our feed forward component helps both on log ppl and Accuracy \marginpar{Intuition?}
%Using dropout to regularize our models helps to prevent overfitting

\subsection{英语成分句法分析}

\begin{table}[t]
\begin{center}
\caption{Transformer可以很好地推广到英语成分句法分析(结果在WSJ 23节上训练而得)}
\label{tab:parsing-results}
\vspace{-2mm}
%\scalebox{1.0}{
\begin{tabular}{c|c|c}
\hline
{\bf Parser}  & {\bf Training} & {\bf WSJ 23 F1} \\ \hline
Vinyals \& Kaiser el al. (2014) \cite{KVparse15}
  & WSJ only, discriminative & 88.3 \\
Petrov et al. (2006) \cite{petrov-EtAl:2006:ACL}
  & WSJ only, discriminative & 90.4 \\
Zhu et al. (2013) \cite{zhu-EtAl:2013:ACL}
  & WSJ only, discriminative & 90.4   \\
Dyer et al. (2016) \cite{dyer-rnng:16}
  & WSJ only, discriminative & 91.7   \\
\specialrule{1pt}{-1pt}{0pt}
Transformer (4 layers)  &  WSJ only, discriminative & 91.3 \\
\specialrule{1pt}{-1pt}{0pt}   
Zhu et al. (2013) \cite{zhu-EtAl:2013:ACL}
  & semi-supervised & 91.3 \\
Huang \& Harper (2009) \cite{huang-harper:2009:EMNLP}
  & semi-supervised & 91.3 \\
McClosky et al. (2006) \cite{mcclosky-etAl:2006:NAACL}
  & semi-supervised & 92.1 \\
Vinyals \& Kaiser el al. (2014) \cite{KVparse15}
  & semi-supervised & 92.1 \\
\specialrule{1pt}{-1pt}{0pt}
Transformer (4 layers)  & semi-supervised & 92.7 \\
\specialrule{1pt}{-1pt}{0pt}   
Luong et al. (2015) \cite{multiseq2seq}
  & multi-task & 93.0   \\
Dyer et al. (2016) \cite{dyer-rnng:16}
  & generative & 93.3   \\
\hline
\end{tabular}
\end{center}
\end{table}

%To evaluate if the Transformer can generalize to other tasks we performed experiments on English constituency parsing. This task presents specific challenges: the output is subject to strong structural constraints and is significantly longer than the input.
%Furthermore, RNN sequence-to-sequence models have not been able to attain state-of-the-art results in small-data regimes \cite{KVparse15}.

为了评估Transformer是否可以推广到其他任务,我们对英语成分句法分析进行了实验。这项任务提出了具体的挑战:输出受制于强大的结构约束,而且比输入要长很多。此外,RNN序列到序列的模型在小数据的情况下还不能达到最先进的结果\cite{KVparse15}。

%We trained a 4-layer transformer with $d_{model} = 1024$ on the Wall Street Journal (WSJ) portion of the Penn Treebank \citep{marcus1993building}, about 40K training sentences. We also trained it in a semi-supervised setting, using the larger high-confidence and BerkleyParser corpora from with approximately 17M sentences \citep{KVparse15}. We used a vocabulary of 16K tokens for the WSJ only setting and a vocabulary of 32K tokens for the semi-supervised setting.

我们用$\dmodel=1024$对Penn Treebank的Wall Street Journal(WSJ)部分进行了4层Transformer的训练,大约有4万个训练句子。我们还在半监督的情况下对其进行了训练,使用了较大的高置信度和BerkleyParser语料库,大约有1700万个句子  \citep{KVparse15}。我们在只使用WSJ的情况下使用了1.6万的词汇,在半监督的情况下使用了3.2万的词汇。

%We performed only a small number of experiments to select the dropout, both attention and residual (section~\ref{sec:reg}), learning rates and beam size on the Section 22 development set, all other parameters remained unchanged from the English-to-German base translation model. During inference, we increased the maximum output length to input length + $300$. We used a beam size of $21$ and $\alpha=0.3$ for both WSJ only and the semi-supervised setting.

我们只进行了少量的实验,在第22节开发集上选择dropout、注意力机制和残差(节~\ref{sec:reg})、学习率和beam大小,所有其他参数与英译德基础翻译模型保持不变。在推理过程中,我们将最大输出长度增加到输入长度+$300$。我们设置beam大小为$21$和$alpha=0.3$,仅在WSJ和半监督下。

%Our results in Table~\ref{tab:parsing-results} show that despite the lack of task-specific tuning our model performs surprisingly well, yielding better results than all previously reported models with the exception of the Recurrent Neural Network Grammar \cite{dyer-rnng:16}.

表\ref{tab:parsing-results}中的结果显示,尽管缺乏特定任务的调整,我们的模型表现得出奇的好,产生的结果比以前报道的所有模型都要好,除了循环神经网络\cite{dyer-rnng:16}。


%In contrast to RNN sequence-to-sequence models \citep{KVparse15}, the Transformer outperforms the BerkeleyParser \cite{petrov-EtAl:2006:ACL} even when training only on the WSJ training set of 40K sentences.

与RNN序列到序列模型\citep{KVparse15}相比,Transformer的表现优于BerkeleyParser\cite{petrov-EtAl:2006:ACL},即使只在WSJ的4万个句子的训练集上进行训练。


\section{结束语}
在这项工作中,我们提出了Transformer,这是第一个完全基于注意力的序列转换模型,用多头自注意力取代了编码器-解码器架构中最常用的循环神经网络。

对于翻译任务,Transformer的训练速度可以显著快于基于循环层或卷积层的架构。在WMT 2014英译德和WMT 2014英译法两个翻译任务上,我们实现了新的技术状态。在前一项任务中,我们的最佳模型甚至优于所有之前已知的组合模型。%我们还通过对英语成分句法分析的实验,提供了我们模型更广泛的适用性的说明。

我们对基于注意力的模型的未来感到兴奋,并计划将其应用于其他任务。我们计划将Transformer扩展到涉及文本以外的输入和输出模式的问题,并研究局部的、受限的注意力机制,以有效地处理图像、音频和视频等大型输入和输出。
让生成过程的顺序性降低是我们的另一个研究目标。

用来训练和评估我们的模型的代码可以在\url{https://github.com/tensorflow/tensor2tensor}上找到。

\paragraph{鸣谢} 我们感谢来自Nal Kalchbrenner和Stephan Gouws的卓有成效的评论,更正和启发。

\bibliographystyle{plain}
%\bibliography{deeplearn}
\begin{thebibliography}{10}

\bibitem{layernorm2016}
Jimmy~Lei Ba, Jamie~Ryan Kiros, and Geoffrey~E Hinton.
\newblock Layer normalization.
\newblock {\em arXiv preprint arXiv:1607.06450}, 2016.

\bibitem{bahdanau2014neural}
Dzmitry Bahdanau, Kyunghyun Cho, and Yoshua Bengio.
\newblock Neural machine translation by jointly learning to align and
  translate.
\newblock {\em CoRR}, abs/1409.0473, 2014.

\bibitem{DBLP:journals/corr/BritzGLL17}
Denny Britz, Anna Goldie, Minh{-}Thang Luong, and Quoc~V. Le.
\newblock Massive exploration of neural machine translation architectures.
\newblock {\em CoRR}, abs/1703.03906, 2017.

\bibitem{cheng2016long}
Jianpeng Cheng, Li~Dong, and Mirella Lapata.
\newblock Long short-term memory-networks for machine reading.
\newblock {\em arXiv preprint arXiv:1601.06733}, 2016.

\bibitem{cho2014learning}
Kyunghyun Cho, Bart van Merrienboer, Caglar Gulcehre, Fethi Bougares, Holger
  Schwenk, and Yoshua Bengio.
\newblock Learning phrase representations using rnn encoder-decoder for
  statistical machine translation.
\newblock {\em CoRR}, abs/1406.1078, 2014.

\bibitem{xception2016}
Francois Chollet.
\newblock Xception: Deep learning with depthwise separable convolutions.
\newblock {\em arXiv preprint arXiv:1610.02357}, 2016.

\bibitem{gruEval14}
Junyoung Chung, {\c{C}}aglar G{\"{u}}l{\c{c}}ehre, Kyunghyun Cho, and Yoshua
  Bengio.
\newblock Empirical evaluation of gated recurrent neural networks on sequence
  modeling.
\newblock {\em CoRR}, abs/1412.3555, 2014.

\bibitem{dyer-rnng:16}
Chris Dyer, Adhiguna Kuncoro, Miguel Ballesteros, and Noah~A. Smith.
\newblock Recurrent neural network grammars.
\newblock In {\em Proc. of NAACL}, 2016.

\bibitem{JonasFaceNet2017}
Jonas Gehring, Michael Auli, David Grangier, Denis Yarats, and Yann~N. Dauphin.
\newblock Convolutional sequence to sequence learning.
\newblock {\em arXiv preprint arXiv:1705.03122v2}, 2017.

\bibitem{graves2013generating}
Alex Graves.
\newblock Generating sequences with recurrent neural networks.
\newblock {\em arXiv preprint arXiv:1308.0850}, 2013.

\bibitem{he2016deep}
Kaiming He, Xiangyu Zhang, Shaoqing Ren, and Jian Sun.
\newblock Deep residual learning for image recognition.
\newblock In {\em Proceedings of the IEEE Conference on Computer Vision and
  Pattern Recognition}, pages 770--778, 2016.

\bibitem{hochreiter2001gradient}
Sepp Hochreiter, Yoshua Bengio, Paolo Frasconi, and J{\"u}rgen Schmidhuber.
\newblock Gradient flow in recurrent nets: the difficulty of learning long-term
  dependencies, 2001.

\bibitem{hochreiter1997}
Sepp Hochreiter and J{\"u}rgen Schmidhuber.
\newblock Long short-term memory.
\newblock {\em Neural computation}, 9(8):1735--1780, 1997.

\bibitem{huang-harper:2009:EMNLP}
Zhongqiang Huang and Mary Harper.
\newblock Self-training {PCFG} grammars with latent annotations across
  languages.
\newblock In {\em Proceedings of the 2009 Conference on Empirical Methods in
  Natural Language Processing}, pages 832--841. ACL, August 2009.

\bibitem{jozefowicz2016exploring}
Rafal Jozefowicz, Oriol Vinyals, Mike Schuster, Noam Shazeer, and Yonghui Wu.
\newblock Exploring the limits of language modeling.
\newblock {\em arXiv preprint arXiv:1602.02410}, 2016.

\bibitem{extendedngpu}
{\L}ukasz Kaiser and Samy Bengio.
\newblock Can active memory replace attention?
\newblock In {\em Advances in Neural Information Processing Systems, ({NIPS})},
  2016.

\bibitem{neural_gpu}
\L{}ukasz Kaiser and Ilya Sutskever.
\newblock Neural {GPU}s learn algorithms.
\newblock In {\em International Conference on Learning Representations
  ({ICLR})}, 2016.

\bibitem{NalBytenet2017}
Nal Kalchbrenner, Lasse Espeholt, Karen Simonyan, Aaron van~den Oord, Alex
  Graves, and Koray Kavukcuoglu.
\newblock Neural machine translation in linear time.
\newblock {\em arXiv preprint arXiv:1610.10099v2}, 2017.

\bibitem{structuredAttentionNetworks}
Yoon Kim, Carl Denton, Luong Hoang, and Alexander~M. Rush.
\newblock Structured attention networks.
\newblock In {\em International Conference on Learning Representations}, 2017.

\bibitem{kingma2014adam}
Diederik Kingma and Jimmy Ba.
\newblock Adam: A method for stochastic optimization.
\newblock In {\em ICLR}, 2015.

\bibitem{Kuchaiev2017Factorization}
Oleksii Kuchaiev and Boris Ginsburg.
\newblock Factorization tricks for {LSTM} networks.
\newblock {\em arXiv preprint arXiv:1703.10722}, 2017.

\bibitem{lin2017structured}
Zhouhan Lin, Minwei Feng, Cicero Nogueira~dos Santos, Mo~Yu, Bing Xiang, Bowen
  Zhou, and Yoshua Bengio.
\newblock A structured self-attentive sentence embedding.
\newblock {\em arXiv preprint arXiv:1703.03130}, 2017.

\bibitem{multiseq2seq}
Minh-Thang Luong, Quoc~V. Le, Ilya Sutskever, Oriol Vinyals, and Lukasz Kaiser.
\newblock Multi-task sequence to sequence learning.
\newblock {\em arXiv preprint arXiv:1511.06114}, 2015.

\bibitem{luong2015effective}
Minh-Thang Luong, Hieu Pham, and Christopher~D Manning.
\newblock Effective approaches to attention-based neural machine translation.
\newblock {\em arXiv preprint arXiv:1508.04025}, 2015.

\bibitem{marcus1993building}
Mitchell~P Marcus, Mary~Ann Marcinkiewicz, and Beatrice Santorini.
\newblock Building a large annotated corpus of english: The penn treebank.
\newblock {\em Computational linguistics}, 19(2):313--330, 1993.

\bibitem{mcclosky-etAl:2006:NAACL}
David McClosky, Eugene Charniak, and Mark Johnson.
\newblock Effective self-training for parsing.
\newblock In {\em Proceedings of the Human Language Technology Conference of
  the NAACL, Main Conference}, pages 152--159. ACL, June 2006.

\bibitem{decomposableAttnModel}
Ankur Parikh, Oscar Täckström, Dipanjan Das, and Jakob Uszkoreit.
\newblock A decomposable attention model.
\newblock In {\em Empirical Methods in Natural Language Processing}, 2016.

\bibitem{paulus2017deep}
Romain Paulus, Caiming Xiong, and Richard Socher.
\newblock A deep reinforced model for abstractive summarization.
\newblock {\em arXiv preprint arXiv:1705.04304}, 2017.

\bibitem{petrov-EtAl:2006:ACL}
Slav Petrov, Leon Barrett, Romain Thibaux, and Dan Klein.
\newblock Learning accurate, compact, and interpretable tree annotation.
\newblock In {\em Proceedings of the 21st International Conference on
  Computational Linguistics and 44th Annual Meeting of the ACL}, pages
  433--440. ACL, July 2006.

\bibitem{press2016using}
Ofir Press and Lior Wolf.
\newblock Using the output embedding to improve language models.
\newblock {\em arXiv preprint arXiv:1608.05859}, 2016.

\bibitem{sennrich2015neural}
Rico Sennrich, Barry Haddow, and Alexandra Birch.
\newblock Neural machine translation of rare words with subword units.
\newblock {\em arXiv preprint arXiv:1508.07909}, 2015.

\bibitem{shazeer2017outrageously}
Noam Shazeer, Azalia Mirhoseini, Krzysztof Maziarz, Andy Davis, Quoc Le,
  Geoffrey Hinton, and Jeff Dean.
\newblock Outrageously large neural networks: The sparsely-gated
  mixture-of-experts layer.
\newblock {\em arXiv preprint arXiv:1701.06538}, 2017.

\bibitem{srivastava2014dropout}
Nitish Srivastava, Geoffrey~E Hinton, Alex Krizhevsky, Ilya Sutskever, and
  Ruslan Salakhutdinov.
\newblock Dropout: a simple way to prevent neural networks from overfitting.
\newblock {\em Journal of Machine Learning Research}, 15(1):1929--1958, 2014.

\bibitem{sukhbaatar2015}
Sainbayar Sukhbaatar, Arthur Szlam, Jason Weston, and Rob Fergus.
\newblock End-to-end memory networks.
\newblock In C.~Cortes, N.~D. Lawrence, D.~D. Lee, M.~Sugiyama, and R.~Garnett,
  editors, {\em Advances in Neural Information Processing Systems 28}, pages
  2440--2448. Curran Associates, Inc., 2015.

\bibitem{sutskever14}
Ilya Sutskever, Oriol Vinyals, and Quoc~VV Le.
\newblock Sequence to sequence learning with neural networks.
\newblock In {\em Advances in Neural Information Processing Systems}, pages
  3104--3112, 2014.

\bibitem{DBLP:journals/corr/SzegedyVISW15}
Christian Szegedy, Vincent Vanhoucke, Sergey Ioffe, Jonathon Shlens, and
  Zbigniew Wojna.
\newblock Rethinking the inception architecture for computer vision.
\newblock {\em CoRR}, abs/1512.00567, 2015.

\bibitem{KVparse15}
{Vinyals {\&} Kaiser}, Koo, Petrov, Sutskever, and Hinton.
\newblock Grammar as a foreign language.
\newblock In {\em Advances in Neural Information Processing Systems}, 2015.

\bibitem{wu2016google}
Yonghui Wu, Mike Schuster, Zhifeng Chen, Quoc~V Le, Mohammad Norouzi, Wolfgang
  Macherey, Maxim Krikun, Yuan Cao, Qin Gao, Klaus Macherey, et~al.
\newblock Google's neural machine translation system: Bridging the gap between
  human and machine translation.
\newblock {\em arXiv preprint arXiv:1609.08144}, 2016.

\bibitem{DBLP:journals/corr/ZhouCWLX16}
Jie Zhou, Ying Cao, Xuguang Wang, Peng Li, and Wei Xu.
\newblock Deep recurrent models with fast-forward connections for neural
  machine translation.
\newblock {\em CoRR}, abs/1606.04199, 2016.

\bibitem{zhu-EtAl:2013:ACL}
Muhua Zhu, Yue Zhang, Wenliang Chen, Min Zhang, and Jingbo Zhu.
\newblock Fast and accurate shift-reduce constituent parsing.
\newblock In {\em Proceedings of the 51st Annual Meeting of the ACL (Volume 1:
  Long Papers)}, pages 434--443. ACL, August 2013.

\end{thebibliography}
%\newpage
\input{visualizations}
%\appendix
%\newpage
%\input{parameter_attention}

%\input{sqrt_d_trick}

\end{document}
